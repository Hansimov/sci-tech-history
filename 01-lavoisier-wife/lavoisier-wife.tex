\documentclass[a4paper]{article}

% Overleaf 链接:https://www.overleaf.com/16655754rygmvynthprv#/63903150/

\usepackage[UTF8]{ctex}
\usepackage[top=2cm, bottom=2cm, left=2cm, right=2cm]{geometry}

\usepackage{amsmath}

\newcommand{\os}[2]{$\overset{\text{#1}}{\text{#2}}$}
\newcommand{\us}[2]{$\underset{\text{#1}}{\text{#2}}$}

\title{拉瓦锡夫人-皮埃尔莱特}
\author{赛同学的古玩店 \\ SVD-陈博衍, Hansimov}

\begin{document}

\maketitle

在中学化学书中,有一幅插图总是在化学书的一开始出现,可每次我们看到这幅图的时候,总是会想“图中到这个男人在干什么,他在向他的右侧观望什么?”

我们的化学教材在这个时候,给我们开了一个小小的玩笑,而这幅画原本的面貌在各大名著的封面图上却屡见不鲜。

这幅画的男主人公想必大家都已经了解了不少,他是十八世纪伟大的化学家,近代化学之父,安托万-洛朗·德·拉瓦锡。而图中的女主角,似乎知道的人就少了。

在介绍女主人公前,我们还应该给拉瓦锡额外添加上一个新的头衔:萝莉控。

1771年的春天,拉瓦锡与自己的单身生活永远的告别了,他与他同事的女儿——玛丽-安娜·皮埃尔莱特一起步入了婚姻的殿堂。但值得一说的是,那时候,玛丽-安娜·皮埃尔莱特只有13岁。但可千万别小看了皮埃尔莱特,她在未来将为拉瓦锡的科学研究工作作出诸多的贡献。正如同她的手轻轻的按在铺着红毯象征着化学世界的桌子上一样,皮埃尔莱特,在化学世界也应当分享她丈夫所获的的荣誉。

不过,话说回来,其实我们并不能苛责拉瓦锡娶了一位芳龄只有13的姑娘。毕竟,婚姻的牵头人事实上是皮埃尔莱特的父亲。他为了躲避五十多岁的老先生向她女儿的求婚,皮埃尔莱特的父亲不得不提前为自己找个女婿,而这时候恰恰他的同事,拉瓦锡与皮埃尔莱特因为自然科学而互相产生了些许的好感。

缘分这种东西的确是玄之又玄,妙不可言,在1771年拉瓦锡与皮埃尔莱特结婚后不久,皮埃尔莱特很快地学会了英语,而拉瓦锡却对英语一窍不通。在此过程中,皮埃尔莱特为拉瓦锡翻译了诸多的英语世界的化学著作。同时,在雅克-路易·大卫的指导下,皮埃尔莱特对工艺绘图技术也颇有成就。她将拉瓦锡所有的实验过程,实验场景一一记录,并诉诸于绘画的技巧完美的展现在了拉瓦锡的著作中。可以说,正是因为这样精妙的绘画与插图,我们几乎能够完全复刻出拉瓦锡著作中所有记录在册的每一个实验。


1778年,在拉瓦锡娶了皮埃尔莱特7年后,拉瓦锡方才真正开始组织自己的科研团队。在这7年中,皮埃尔莱特作为拉瓦锡的实验助手、科研秘书,同时也作为拉瓦锡的夫人为拉瓦锡召集或举办一些上流社会的艺术沙龙。在现在看来,皮埃尔莱特在拉瓦锡的所有著作中作为第二作者出现都再也正常不过了,可因为时代的局限,这一切的功勋,皮埃尔莱特只能通过她的智慧悄悄地隐藏在她绘制的精美插图中。

至于图中所绘制的若干仪器与拉瓦锡所做的实验,那就是下一个故事了。


\end{document}