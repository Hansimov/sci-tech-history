\documentclass[a4paper]{article}

% Overleaf 链接:https://www.overleaf.com/16655754rygmvynthprv#/63903150/

\usepackage[UTF8]{ctex}
\usepackage[top=2cm, bottom=2cm, left=2cm, right=2cm]{geometry}

\title{化学家背后的女人:玻尔兹}
\author{赛同学的古玩店 \\ SVD-陈博衍, Hansimov}

\begin{document}

\maketitle

% 在中学化学书中,有一幅插图总是在化学书的一开始出现,可每次我们看到这幅图的时候,总是会想“图中到这个男人在干什么,他在向他的右侧观望什么?”
细心的同学可能会发现,中学化学书中有这样一幅插图:画面中,一位男子手里拿着羽毛笔,看向自己的右侧。有人一定会好奇,他到底在看什么呢?
% 我们的化学教材在这个时候,给我们开了一个小小的玩笑,
其实啊,我们的教材中只放了原图的一半,完整的图片是这样的。【放全图】
% 而这幅画原本的面貌在各大名著的封面图上却屡见不鲜。
这幅画曾经出现在不少名著的封面上,虽然似乎书的内容和拉瓦锡并没有直接的关系。这也许算得上是最早的“开局一张图,故事全靠编”了。

% 关于这幅画的更多介绍:https://en.wikipedia.org/wiki/Portrait_of_Antoine-Laurent_Lavoisier_and_his_Wife#History

% 这幅画的男主人公想必大家都已经了解了不少,他是十八世纪伟大的化学家,近代化学之父,安托万-洛朗·德·拉瓦锡
画中的男子,就是是 18 世纪伟大的化学家,近代化学之父,
\os{Antoine}{安托万}\os{}{-}\os{Laurent}{洛朗}·\os{de}{德}·\os{Lavoisier}{拉瓦锡} (1743.08.26 – 1794.05.08)。
% 而图中的女主角,似乎知道的人就少了。
而对画中的女子,大部分人可能就知之甚少了。

% 在介绍女主人公前,我们还应该给拉瓦锡额外添加上一个新的头衔:萝莉控。

% 1771年的春天,
故事要从1771年的春天开始说起。在那个美好的春天,拉瓦锡结识了一个女孩,
名叫 \os{Marie}{玛丽}\os{}{-}\os{Anne}{安妮}·\os{Pierrette}{皮埃尔丽特}·\os{Paulze}{玻尔兹} (1758.01.20 – 1836.02.10),
% 但值得一说的是,那时候,皮埃尔莱特只有13岁。
彼时玻尔兹才 13 岁,和28岁的拉瓦锡足足相差了15岁,所以拉瓦锡应该算是萝莉控界的前辈了。

% 不过,话说回来,其实我们并不能苛责拉瓦锡娶了一位芳龄只有13的姑娘。毕竟,婚姻的牵头人事实上是皮埃尔莱特的父亲。他为了躲避五十多岁的老先生向她女儿的求婚,皮埃尔莱特的父亲不得不提前为自己找个女婿,而这时候恰恰他的同事,拉瓦锡与皮埃尔莱特因为自然科学而互相产生了些许的好感。
在这段跨越年龄的爱恋背后,还有一个小插曲。他们二人订婚之前,有一位五十岁的伯爵曾向玻尔兹求婚。玻尔兹的父亲,\os{Jacques}{雅克}·\os{玻尔兹}{Paulze},为了女儿的前途,不得不提前为自己找个女婿。恰好在这时候,他的一位年轻同事,也就是拉瓦锡,和玻尔兹在自然科学上兴趣相投,互生情愫。

于是,在同一年的冬天,拉瓦锡永远告别了自己的单身生活,和玻尔兹步入了婚姻的殿堂。
% 但可千万别小看了皮埃尔莱特,她在未来将为拉瓦锡的科学研究工作作出诸多的贡献。正如同她的手轻轻的按在铺着红毯象征着化学世界的桌子上一样,皮埃尔莱特,在化学世界也应当分享她丈夫所获的的荣誉。
而玻尔兹将会在未来的日子里,为拉瓦锡的科学事业提供极大的支持。是的,在拉瓦锡的伟大成果背后,站着一位充满智慧的女人。

% 缘分这种东西的确是玄之又玄,妙不可言,在1771年拉瓦锡与皮埃尔莱特结婚后不久,
婚后不久,玻尔兹很快学会了英语,而拉瓦锡则对英语一窍不通。
于是,玻尔兹为拉瓦锡翻译了许多英语世界的化学著作。

同时,在\os{Jacques}{雅克}\os{}{-}\os{Louis}{路易}\os{David}{大卫}的指导下,玻尔兹在工艺绘图上也颇有造诣。她将拉瓦锡的实验场景一一记录下来,
% 并诉诸于绘画的技巧完美的展现在了拉瓦锡的著作中。可以说,正是因为这样精妙的绘画与插图,我们几乎能够完全复刻出拉瓦锡著作中所有记录在册的每一个实验。
绘制成精美的插图,放在拉瓦锡的著作中。正是有了这些插图,我们几乎能够将拉瓦锡著作中记录的每一个实验,都完整地复现出来。

% 在现在看来,皮埃尔莱特在拉瓦锡的所有著作中作为第二作者出现都再也正常不过了,
如果放到今天,玻尔兹的名字一定会以第二作者的身份,写在拉瓦锡大部分著作的封面上。
% 可因为时代的局限,这一切的功勋,玻尔兹只能通过她的智慧悄悄地隐藏在她绘制的精美插图中。
可惜因为时代的局限,玻尔兹只好将自己的贡献,默默地留存在那些亲手绘制的精美插图中了。

在婚后的许多年里,玻尔兹既担任了拉瓦锡的实验助手和科研秘书,同时也以拉瓦锡夫人的身份,为丈夫筹备和举办了不少上流社会的艺术沙龙。

直到1778年,在娶了玻尔兹7年后,拉瓦锡才真正开始组织起属于自己的科研团队。

而那又是之后的故事了。

% 至于图中所绘制的若干仪器与拉瓦锡所做的实验,那就是下一个故事了。

\end{document}